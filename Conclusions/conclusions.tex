\chapter{Conclusions}
\label{chap:conclusions}
\section{Summary}

Numerical models of weather and climate are high-performance applications for the world's largest supercomputers, producing very large amounts of data. 
While 32 and 64-bit floating-point numbers are the only widely available number formats for scientific computing, this thesis investigates the potential
for lower precision computations and data compression for weather and climate modelling. An information-preserving compression for climate data
is derived, presented and discussed that distinguishes between the real and false information in data.



Low-precision climate computing can accelerate simulations
such that freed resources can be reinvested into improving 


\section{Discussion}