\chapter{A 16-bit shallow water model}
\label{chap:shallow_water}

%% CONTRIBUTION
\paragraph{Contributions} This chapter is largely based on the following publications \footnote{with the following author contributions.
Conceptualisation: MK, PDD. Data curation: MK. Formal Analysis: MK. Methodology: MK. Visualisation: MK. Writing – original draft:
MK. Writing – review and editing: MK, PDD, TNP. The contributions of Peter and Tim are highly appreciated.}

\vspace{\baselineskip}
\indent M Klöwer, PD Düben, and TN Palmer, 2019. \emph{Posits as an alternative to floats for weather and climate models}, \textbf{CoNGA'19: Proceedings of the Conference for Next Generation Arithmetic}, Singapore, doi:\texttt{10.1145/3316279.3316281}.

\indent M Klöwer, PD Düben, and TN Palmer, 2020. \emph{Number Formats, Error Mitigation, and Scope for 16-Bit Arithmetics in Weather and Climate Modeling Analyzed With a Shallow Water Model}, \textbf{Journal of Advances in Modeling Earth Systems}, doi:\texttt{10.1029/2020MS002246}.
\vspace{\baselineskip}

\section{Introduction}

... which was often debated as a step in the wrong direction.

\section{Methods}
\subsection{The shallow water model}
\subsection{Scaling and reordering the shallow water equations}
\subsection{Mixed precision}
\subsection{Compensated time integration}
\subsection{Reduced precision communication for distributed computing}
\subsection{A 16-bit semi-Lagrangian advection scheme}

\section{Impact of low-precision on the physics}
\subsection{Error growth}
\subsection{Mean and variability}
\subsection{Geostrophy}
\subsection{Gravity waves}
\subsection{Mass and tracer conservation}

\section{Discussion}