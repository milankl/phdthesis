\begin{abstracts}

Progress towards more reliable weather and climate forecasts is limited by the resolution of numerical models and the complexity of simulated
processes. Performance is therefore a major bottleneck and current models are not computationally efficient. High precision calculations are
unnecessary, despite being the standard, given the uncertainties in the climate system and the errors from discretisation, data assimilation
and unresolved climate processes. In this thesis, we advance several aspects of low-precision climate computing to preserve information despite
fewer bits: An information-preserving compression is developed that distinguishes between real and false information to reduce the very large
volume of climate data produced by numerical models, while minimising information loss. The bitwise real information content estimates the
minimum required precision in climate data, which depends on the variable and is lower than the standard precision-levels of floating-point numbers. 
The impact of rounding errors introduced by different low-precision arithmetics with deterministic or stochastic rounding modes is analysed
in chaotic dynamical systems. Standard floating-point numbers are not the best number format for weather and climate simulations. However,
alternatives, such as posits, exist, but it is unclear whether the large effort needed to develop the respective hardware for future supercomputers
is justified given the moderate advantage they provide in our applications. A much more central issue towards 16-bit climate models is the design
of low precision-resilient algorithms. A naive transition to 16 bits either fails or was found to cause issues like amplified gravity waves,
a change in geostrophy or rounding errors that grow as quickly as discretisation errors. However, many of these issues are found to be
preventable with techniques such as scaling or a compensated time integration. Combining techniques, we develop a 16-bit fluid circulation
model that approaches 4x speedups on Fujitsu's A64FX processor compared to 64 bits, despite minimal rounding errors. The result of this
thesis show that there is little reason to assume that 16-bit weather and climate models are not possible. While the design of models to
compute and output only the bitwise real information is challenging, it will be a major step towards computationally efficient digital twins
of the Earth's climate system.


\end{abstracts}
